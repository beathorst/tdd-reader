\documentclass[a4paper,12pt]{scrartcl}

%----------------------------------------------------------------------------------
% Pakete und Parameter
%----------------------------------------------------------------------------------

% Pound-Zeichen statt $ in Code-Kommentaren fixen
\usepackage[T1]{fontenc}

% Input-Encoding für UTF-8
\usepackage[utf8]{inputenc}

% Mehr von der Seitenbreite nutzen
\usepackage{a4wide}

% Grafikpaket
\usepackage{color}
\usepackage[pdftex]{graphicx}

% Absätze werden nicht eingezogen, sondern vertikal abgesetzt
\setlength{\parindent}{0mm}
\addtolength{\parskip}{0.4em}

% Palatino und Helvetica statt Computer Modern als Standard-Fonts
\usepackage{palatino}

% lesbare Verweise
\usepackage[pdftex,plainpages=false,pdfpagelabels]{hyperref}

% nette URLs
\usepackage{url}

% für Boxen etc.
\usepackage{framed}
\definecolor{shadecolor}{rgb}{0.8,0.8,0.8}

% Anführungszeichen sprachabhängig machen
\usepackage[babel]{csquotes}


%----------------------------------------------------------------------------------
% Seitenlayout
%----------------------------------------------------------------------------------

% zweispaltiges Layout möglich machen
\usepackage{multicol}

% keine Seitenzahlen
\pagenumbering{gobble}
\pagestyle{empty}

% Seitenlayout
\topmargin0mm
\footskip0mm

% Descriptions ohne Einzug
\renewenvironment{description}[1][0pt]
{\list{}{
    \labelwidth=0pt \leftmargin=#1
     \let\makelabel\descriptionlabel
  }
}
{\endlist}

\raggedbottom
\raggedright

\newcommand{\fett}[1]{\textsf{\textbf{#1}}}

\newcommand{\answerspace}{\vspace{2.8em}}

\usepackage[ngerman]{babel}
\selectlanguage{ngerman}


\begin{document}
\raggedbottom

\section*{Übungen zu TDD mit TYPO3 CMS 7.6}

\begin{enumerate}
  \item Installiert euch die Extensions \texttt{phpunit} und \texttt{extension\_builder}.
  
  \item Lasst die Tests der \texttt{phpunit}-Extension im Backend laufen.
  
  \item Legt mit dem Extension-Builder eine leere Extension \texttt{stock} an. Legt zwei leere Models \texttt{Product} und \texttt{Color} an und löscht die automatisch generierten Tests wieder.
  
  \item Öffnet die Extension in eurer IDE und bindet den TYPO3-Source sowie die PHPUnit-Extension ein (oder die komplette TYPO3-Installation).
  
  \item Schreibt ein \texttt{Test/Unit/MathTest.php} einen Test, der überprüft, dass $1 + 1 = 2$ ist.\footnote{Es ist gut, das ab und an zu überprüfen, weil wir ein echtes Problem bekommen, falls das irgendwann nicht mehr stimmen sollte.} Führt die Tests aus.
  
  \item Schreibt einen Test für die beiden Model-Klasse, dass sie instanziiert werden können und Unterklassen der Model-Basisklasse sind.
  
  \item Fügt (testgetrieben) ein Feld \texttt{\$title} zu beiden Models hinzu. Werft eine \texttt{UnexpecteValueException}, falls bei setTitle ein leerer String übergeben wird.
  
  \item Legt (testgetrieben) eine 1:n-Assoziation \texttt{Product.colors} an (inklusive aller nötigen Methoden).
  
  \item Schreibt (testgetrieben) eine Methode \texttt{Product.getFullTitle}, die den Title inklusive aller Farben zurückgibt.
  
  \item Schreibt ein \texttt{ProductRepository} und testet, dass ihr es instanziieren könnt und dass es eine Unterklasse der Repository-Basisklasse ist. Ihr benötigt dafür wahrscheinlich ein gemockten \texttt{ObjectManager}. Schaut euch dafür den Konstruktor der Basisklasse an.
  
  \item Schreibt einen \texttt{Service/ImportService.php}, der Produkt-Titel aus einer Textdatei liest (eine pro Zeile) und diese im Repository neu anlegt. Der Service braucht nicht zu überprüfen, ob es schon Produkte mit demselben Titel gibt.
 
\end{enumerate}


\end{document}
