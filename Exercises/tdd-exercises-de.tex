\documentclass[a4paper,12pt]{scrartcl}

%----------------------------------------------------------------------------------
% Pakete und Parameter
%----------------------------------------------------------------------------------

% Pound-Zeichen statt $ in Code-Kommentaren fixen
\usepackage[T1]{fontenc}

% Input-Encoding für UTF-8
\usepackage[utf8]{inputenc}

% Mehr von der Seitenbreite nutzen
\usepackage{a4wide}

% Grafikpaket
\usepackage{color}
\usepackage[pdftex]{graphicx}

% Absätze werden nicht eingezogen, sondern vertikal abgesetzt
\setlength{\parindent}{0mm}
\addtolength{\parskip}{0.4em}

% Palatino und Helvetica statt Computer Modern als Standard-Fonts
\usepackage{palatino}

% lesbare Verweise
\usepackage[pdftex,plainpages=false,pdfpagelabels]{hyperref}

% nette URLs
\usepackage{url}

% für Boxen etc.
\usepackage{framed}
\definecolor{shadecolor}{rgb}{0.8,0.8,0.8}

% Anführungszeichen sprachabhängig machen
\usepackage[babel]{csquotes}


%----------------------------------------------------------------------------------
% Seitenlayout
%----------------------------------------------------------------------------------

% zweispaltiges Layout möglich machen
\usepackage{multicol}

% keine Seitenzahlen
\pagenumbering{gobble}
\pagestyle{empty}

% Seitenlayout
\topmargin0mm
\footskip0mm

% Descriptions ohne Einzug
\renewenvironment{description}[1][0pt]
{\list{}{
    \labelwidth=0pt \leftmargin=#1
     \let\makelabel\descriptionlabel
  }
}
{\endlist}

\raggedbottom
\raggedright

\newcommand{\fett}[1]{\textsf{\textbf{#1}}}

\newcommand{\answerspace}{\vspace{2.8em}}

\usepackage[ngerman]{babel}
\selectlanguage{ngerman}


\begin{document}
\raggedbottom

\section{Übungen zu TDD mit PHP und TYPO3 CMS 7.6/8.7}

\subsection{TDD mit Composer (ohne TYPO3)}
\begin{enumerate}
  \item Klont euch das TDD-Seed-Projekt\footnote{\url{https://github.com/oliverklee/tdd-seed}, alternativ per Composer als \texttt{oliverklee/tdd-seed}}.
  \item Führt ein \texttt{composer install} aus.
  \item Richtet PHPUnit in PhpStorm ein und lasst die Tests des Projektes in PhpStorm laufen.
  \item Lasst die Tests über die Kommandozeile laufen.
  \item Wechselt vom \texttt{master}-Branch auf den Branch {feature/coffee}. Führt alle Tests aus.
  \item Schreibt in \texttt{Test/Unit/MathTest.php} einen Test, der überprüft, dass $1 + 1 = 2$ ist.\footnote{Es ist gut, das ab und an zu überprüfen, weil wir ein echtes Problem bekommen, falls das irgendwann nicht mehr stimmen sollte.} Führt die Tests aus.
  \item Nehmt euch ein Blatt Papier und schreibt \emph{Testliste} darauf. Das ist eure Testliste, auf die ihr notiert, welche Tests ihr schreiben möchtet.
  \item Legt ein \texttt{SizeOption}-Model an. Testet, dass es instanziiert werden kann.
  \item Fügt (testgetrieben) ein Feld Integer-Feld \texttt{SizeOption.milliliters}  hinzu. Werft eine \texttt{UnexpecteValueException}, falls beim Setter 0 oder ein negativer Wert übergeben wird.
  \item Fügt testgetrieben ein Boolean-Feld \texttt{SizeOption.isIncludedInPrice} mit Getter und Setter hinzu.
  \item Legt (testgetrieben) eine 1:n-Assoziation \texttt{CoffeeBeverage.sizeOptions} an (inklusive aller nötigen Methoden).
  \item Schreibt (testgetrieben) eine Methode \texttt{CoffeeBeverage.getPriceTag}, die als einen String die verfügbaren Größen zurückgibt inklusive der Information, ob die Größe im Preis enthalten ist oder extra bezahlt werden muss.
\end{enumerate}
\pagebreak

\subsection{TDD mit TYPO3 CMS}
\begin{enumerate}
  \item Klont euch die tea-Extension\footnote{\url{https://github.com/oliverklee/tea}, alternativ per Composer als \texttt{oliverklee/tea}}.
  \item Richtet PHPUnit in PhpStorm ein und lasst die Unit-Tests der Extension in PhpStorm laufen.
  \item Richtet die Datenbank-Zugangsdaten für die Tests in PhpStorm ein und lasst die funktionalen Tests laufen.
  \item Legt euch eine neue Testliste an.
  \item Schreibt testgetrieben (mit Unit-Tests) beim Testimonial-Controller eine \texttt{showAction}.
  \item Schreibt testgetrieben (mit Functional-Tests) \texttt{TestimonialRepository.findOneLatest}, die das zuletzt hinzugefügte Testimonial findet.
  \item Schreibt testgetrieben (mit Unit-Tests) beim Testimonial-Controller eine \texttt{showLatestAction}.
  \item Schreibt einen \texttt{Service/TeaBeverageImportService.php}, der \texttt{TeaBeverage}-Texte aus einer Textdatei liest (eins pro Zeile) und diese im Repository neu anlegt. Der Service braucht nicht zu überprüfen, ob es schon Models mit demselben Titel gibt.
\end{enumerate}

\end{document}
