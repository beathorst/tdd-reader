\documentclass[a4paper,10pt,headsepline]{scrartcl}
\usepackage[utf8]{inputenc}
\usepackage{a4wide,ae,hyperref,url}

\newcommand{\fett}[1]{\textsf{\textbf{#1}}}
\raggedbottom
\setcounter{secnumdepth}{3}

% Provide syntax highlighting for code
\newminted{php}{bgcolor=lightgrey,startinline=true,linenos=true}
\newminted{json}{bgcolor=lightgrey,startinline=false,linenos=true}
\newminted{bash}{bgcolor=lightgrey,startinline=false,linenos=true}

% Set of colors which look nice together
\definecolor{orange_light}{rgb}{1,0.8,0.4}
\definecolor{orange_med}{rgb}{0.753,0.62,0.373}
\definecolor{orange_dark}{rgb}{0.506,0.412,0.251}

\definecolor{green_light}{rgb}{0.8,1,0.4}
\definecolor{green_med}{rgb}{0.635,0.745,0.376}
\definecolor{green_dark}{rgb}{0.435,0.498,0.255}

\definecolor{blue_light}{rgb}{0.4,0.8,1}
\definecolor{blue_med}{rgb}{0.365,0.624,0.749}
\definecolor{blue_dark}{rgb}{0.251,0.42,0.502}

\definecolor{blue_cs3}{rgb}{0,.4,.8}

\definecolor{pink_light}{rgb}{1,0.435,0.812}
\definecolor{pink_med}{rgb}{0.745,0.38,0.62}
\definecolor{pink_dark}{rgb}{0.498,0.255,0.416}

\definecolor{yellow_light}{rgb}{1,1,0.4}
\definecolor{yellow_med}{rgb}{0.757,0.745,0.373}
\definecolor{yellow_dark}{rgb}{0.506,0.49,0.251}

\definecolor{lightgrey}{rgb}{.9,.9,.9}

% blue (for URLs)
\definecolor{blue}{rgb}{0,0,1}


\usepackage[american]{babel}
\selectlanguage{american}

\hypersetup{
 pdfauthor={Oliver Klee, typo3-coding@oliverklee.de},
 pdftitle={Test-driven development with PHPUnit (with and without TYPO3)},
 pdfsubject={Handout for Oliver Klee's workshops on test-driven development with PHPUnit (with and without TYPO3).},
 pdfkeywords={TDD, unit testing, PHPUnit, workshop, seminar, PHP}
}

\author{
  Oliver Klee | \texttt{typo3-coding@oliverklee.de} | \texttt{@oliklee} \\
  \url{https://github.com/oliverklee/tdd-reader}
}

\date{Version 2.0.2, \today, for TYPO3 PHP >= 5.6 and CMS 7.6}

\title{
  Cheatsheet for test-driven development with PHPUnit
}

\begin{document}

\maketitle

\section*{License}

This handout is licensed under a \emph{Creative Commons} license, in this case under an \emph{Attribution-ShareAlike~4.0 (CC BY-SA 4.0)}. This means that you can use, edit and distribute this handout (even commercially) under the following conditions:

\begin{description}
  \item[Attribution.] You need to give credit to the author (me) by listing my name (Oliver Klee). If you also list the source\footnote{\url{https://github.com/oliverklee/tdd-reader}}, that would be nice. And if you want to make me happy, please drop me an e-mail if you use this document.
  \item[ShareAlike.] If you edit or change this document or use it as a basis for some other document, you must use the same license for the resulting document.
  \item[Name the license.] If you distribute this document, you'll need to mention or enclose the license.
\end{description}

You can find a more comprehensive version of this license online.\footnote{\url{http://creativecommons.org/licenses/by-sa/4.0/}}


\pagebreak
\tableofcontents

\pagebreak
\section{A note about TYPO3 CMS 8.7}

The TYPO3-specific parts of this handout apply to TYPO3 CMS 7.6. For version 8.7, there are a few differences:

\begin{itemize}
  \item Use the TYPO3 testing framework\footnote{\url{https://github.com/TYPO3/testing-framework}} instead of the testing framework that comes with the PHPUnit extension.
  \item The test base class now comes from the TYPO3 testing framework, not from the TYPO3 Core.
  \item As PHP 5.6 is not supported anymore with TYPO3 CMS 8.7, you can use PHPUnit 6 (which requires PHP 7).
\end{itemize}



\section{File and class naming}

\subsection{File names}

\begin{tabular}{|l|l|}
  \hline
  \fett{Production code file name} & \fett{Test file name} \\
  \hline
  \texttt{Classes/Domain/Model/Shoe.php} & \texttt{Tests/Unit/Domain/Model/ShoeTest.php} \\
  \hline
  \texttt{Classes/Service/BaristaService.php} & \texttt{Tests/Unit/Service/BaristaServiceTest.php} \\
  \hline
\end{tabular}


\subsection{Class names}

\small
\begin{tabular}{|l|l|}
  \hline
  \fett{Production code class name} & \fett{Test class name} \\
  \hline
  \texttt{Shoes\textbackslash Shop\textbackslash Domain\textbackslash Model\textbackslash Shoe} & \texttt{Shoes\textbackslash Shop\textbackslash Tests\textbackslash Unit\textbackslash Domain\textbackslash Model\textbackslash ShoeTest} \\
  \hline
  \texttt{Shoes\textbackslash Shop\textbackslash Service\textbackslash BaristaService} & \texttt{Shoes\textbackslash Shop\textbackslash Tests\textbackslash Unit\textbackslash Service\textbackslash BaristaServiceTest} \\
  \hline
\end{tabular}
\normalsize


\pagebreak
\section{Test class structure}

\subsection{Extbase extensions}

There's an example project (the tea example) for this on GitHub:\\
\url{https://github.com/oliverklee/ext_tea}\\

\begin{phpcode}
namespace OliverKlee\Shop\Tests\Unit\Domain\Model;

use OliverKlee\Shop\Domain\Model\Article;

class ArticleTest extends \TYPO3\CMS\Core\Tests\UnitTestCase {
    /**
     * @var Article;
     */
    protected $subject = null;

    protected function setUp()
    {
        $this->subject = new Article;
        $this->subject->initializeObject();
    }

    /**
     * @test
     */
    public function getNameInitiallyReturnsEmptyString()
    {
        self::assertSame('', $this->subject->getName());
    }

    /**
     * @test
     */
    public function setNameSetsName()
    {
        $name = 'foo bar';

        $this->subject->setName($name);

        self::assertSame($name, $this->subject->getName());
    }

    // ...
}
\end{phpcode}

\subsection{Non-extbase extensions}

\begin{phpcode}
class AttachmentTest extends \Tx_Phpunit_TestCase {
    /**
     * @var \Tx_Oelib_Attachment
     */
    protected $subject = null;

    protected function setUp()
    {
        $this->subject = new \Tx_Oelib_Attachment();
    }

    /**
     * @test
     */
    public function getFileNameInitiallyReturnsAnEmptyString()
    {
        self::assertSame('', $this->subject->getFileName());
    }

    /**
     * @test
     */
    public function getFileNameWithFileNameSetReturnsFileName()
    {
        $fileName = 'test.txt';

        $this->subject->setFileName($fileName);

        self::assertSame($fileName, $this->subject->getFileName());
    }

    /**
     * @test
     * @expectedException \UnexpectedValueException
     */
    public function setFileNameWithEmptyFileNameThrowsException()
    {
        $this->subject->setFileName('');
    }

    // ...
}
\end{phpcode}

\subsection{Non-TYPO3 PHP projects with Composer}

There is an empty starter project for this on GitHub:\\
\url{https://github.com/oliverklee/tdd-seed}


\subsubsection{composer.json}

This setup installs PHPUnit and vfsStream \fett{for PHP up to 5.6}:\\

\begin{jsoncode}
{
    "require-dev": {
        "phpunit/phpunit": "^5.7.19",
        "mikey179/vfsStream": "^1.6.4"
    },
    "autoload": {
        "psr-4": {
            "..."
        }
    },
    "autoload-dev": {
        "psr-4": {
            "..."
        }
    }
}
\end{jsoncode}

This setup installs PHPUnit and vfsStream \fett{for PHP 6}:\\

\begin{jsoncode}
{
    "require-dev": {
        "phpunit/phpunit": "^6.0.13",
        "mikey179/vfsStream": "^1.6.4"
    },
    "autoload": {
        "psr-4": {
            "..."
        }
    },
    "autoload-dev": {
        "psr-4": {
            "..."
        }
    }
}
\end{jsoncode}

\subsubsection{Test case}
\begin{phpcode}
namespace OliverKlee\Books\Tests\Unit\Domain\Model;

use OliverKlee\Books\Domain\Model;

class BookTest extends \PHPUnit_Framework_TestCase {
    /**
     * @var Book
     */
    protected $subject = null;

    protected function setUp()
    {
        $this->subject = new Book();
    }

    /**
     * @test
     */
    public function getTitleInitiallyReturnsEmptyString()
    {
        self::assertSame('', $this->subject->getTitle());
    }

    /**
     * @test
     */
    public function setTitleSetsTitle()
    {
        $title = 'foo bar';

        $this->subject->setTitle($title);

        self::assertSame('foo bar', $this->subject->getTitle());
    }
}
\end{phpcode}


\pagebreak
\section{Executing the tests}

\subsection{Non-TYPO3 projects}

\subsubsection{On the command line}

\begin{bashcode}
vendor/bin/phpunit Test/
\end{bashcode}

\subsubsection{Within PhpStorm}
\begin{enumerate}
  \item Settings > Languages \& Frameworks > PHP > PHPUnit
  \item PHPUnit library > Use Composer autoloader
  \item PHPUnit library > Path to script: \texttt{vendor/autoload}
  \item OK
  \item right-click on the \texttt{Tests/} folder (or any test file or folder)
  \item Run 'Tests'
\end{enumerate}


\subsection{TYPO3 extensions}

\subsubsection{Within PhpStorm}

\paragraph{For an existing TYPO3 installation in Composer mode:}

This will also load all existing extensions (including the PHPUnit extension), making it possible to use the features of the PHPUnit extension.

\begin{enumerate}
  \item Settings > Languages \& Frameworks > PHP > PHPUnit
  \item PHPUnit library > Use Composer autoloader
  \item PHPUnit library > Path to script: \texttt{vendor/autoload} within the TYPO3 document root
  \item Test runner > Default configuration file: \texttt{typo3/sysext/core/Build/UnitTests.xml} within the TYPO3 document root
  \item OK
  \item Run > Edit Configurations
  \item Defaults > PHPUnit
  \item Command Line > Environment variables
  \item add two variables:
    \begin{itemize}
      \item \texttt{TYPO3\_CONTENT = Development}
      \item \texttt{TYPO3\_PATH\_WEB =} the absolute path to the TYPO3 document root (without the trailing slash)
    \end{itemize}
  \item right-click on the \texttt{Tests/} folder (or any test file or folder)
  \item Run 'Tests'
\end{enumerate}


\paragraph{For an existing TYPO3 installation in classic mode (non-Composer mode):}

In this case, you will not be able to autoload any classe from other extensions, i.\,e., you will not be able to use any features of the PHPUnit extension (or of any other extension dependencies).

\begin{enumerate}
  \item If you have downloaded the TYPO3 source via git instead of as a TAR package, you'll need to do a \texttt{composer install} in the TYPO3 source directory.
  \item Settings > Languages \& Frameworks > PHP > PHPUnit
  \item PHPUnit library > Use Composer autoloader
  \item PHPUnit library > Path to script: \texttt{vendor/autoload} within the TYPO3 source
  \item Test runner > Default configuration file: \texttt{typo3/sysext/core/Build/UnitTests.xml} within the TYPO3 document root
  \item OK
  \item Run > Edit Configurations
  \item Defaults > PHPUnit
  \item Command Line > Environment variables
  \item add two variables:
    \begin{itemize}
      \item \texttt{TYPO3\_CONTENT = Development}
      \item \texttt{TYPO3\_PATH\_WEB =} the absolute path to the TYPO3 document root (without the trailing slash)
    \end{itemize}
  \item right-click on the \texttt{Tests/} folder (or any test file or folder)
  \item Run 'Tests'
\end{enumerate}


\paragraph{Without using an existing TYPO3 installation}

This is the approach used in the TYPO3 extension skeleton\footnote{\url{https://github.com/helhum/ext_scaffold}} by Helmut Hummel and Nicole Cordes.

Add the following sections the \texttt{composer.json} of your extension:

\small
\begin{jsoncode}
"require": {
  "typo3/cms": "~7.6.0"
},
"require-dev": {
  "namelesscoder/typo3-repository-client": "^1.2",
  "nimut/testing-framework": "^1.0",
  "mikey179/vfsStream": "^1.4",
  "phpunit/phpunit": "^4.7 || ^5.0"
},
"config": {
  "vendor-dir": ".Build/vendor",
  "bin-dir": ".Build/bin"
},
"scripts": {
  "post-autoload-dump": [
    "mkdir -p .Build/Web/typo3conf/ext/",
    "[ -L .Build/Web/typo3conf/ext/tea ] || ln -snvf ../../../../. .Build/Web/typo3conf/ext/tea"
  ]
},
"extra": {
  "typo3/cms": {
    "cms-package-dir": "{$vendor-dir}/typo3/cms",
    "web-dir": ".Build/Web"
  }
}
\end{jsoncode}
\normalsize

You'll need to replace \texttt{tea} in line 18 with the key of your extension.

If you'd like to use other extensions that are available from the TER (e.\,g., the PHPUnit extension), you'll need to add (or merge) these sections to your \texttt{composer.json}:

\begin{jsoncode}
"repositories": [
{
  "type": "composer",
  "url": "https://composer.typo3.org/"
}
],
"require-dev": {
  "typo3-ter/phpunit": "*"
},
\end{jsoncode}

Then do the following in PhpStorm:
\begin{enumerate}
  \item Settings > Languages \& Frameworks > PHP > PHPUnit
  \item PHPUnit library > Use Composer autoloader
  \item PHPUnit library > Path to script:
    \begin{itemize}
      \item click on the \emph{Show hidden files and directories} button
      \texttt{.Build/vendor/autoload.php} within the extension directory
    \end{itemize}
  \item Test runner > Default configuration file: \texttt{typo3/sysext/core/Build/UnitTests.xml} within the TYPO3 source
  \item OK
  \item Run > Edit Configurations
  \item Defaults > PHPUnit
  \item Command Line > Environment variables
  \item add two variables:
    \begin{itemize}
      \item \texttt{TYPO3\_CONTENT = Development}
      \item \texttt{TYPO3\_PATH\_WEB =} the absolute path to \texttt{.Build/Web} folder in your extension directory (without the trailing slash)
    \end{itemize}
  \item right-click on the \texttt{Tests/} folder (or any test file or folder)
  \item Run 'Tests'
\end{enumerate}



\subsubsection{Using the PHPUnit back-end module}

This works for TYPO3 installations both in Composer mode and in classic mode.

This will load all installed extensions (including the PHPUnit extension), making it possible to use the features of the PHPUnit extension.

However, this will also execute the tests in the current back-end context, making the tests very brittle. This works fine for most unit tests of extensions, but will not work for functional tests.

\begin{enumerate}
  \item Admin > PHPUnit
\end{enumerate}



\pagebreak
\section{Mocks}

\subsection{Why mock?}
\begin{itemize}
  \item to ``disable'' a method (to not write to the DB, or to not launch a cruise missile) and return null
  \item to have a method redurn a particular return value or throw an exception
  \item to test that a method gets called in a certain way
\end{itemize}

\subsection{Tools for mocking}

\paragraph{Prophecy:} The recommended, state-of the art, easy-to-use mocking framework. It cannot create partial mocks, though.\footnote{Prophecy cheatsheet:\\ \url{https://github.com/oliverklee/tdd-reader/blob/master/AdditionalDocuments/prophecy-cheatsheet.pdf}}
\paragraph{PHPUnit mocks:} The old way of creating mocks. Creating mocks is a bit unwieldy, but it can create partial mocks.\footnote{PHPUnit mocking cheatsheet:\\ \url{https://github.com/oliverklee/tdd-reader/blob/master/AdditionalDocuments/mocking-cheatsheet.pdf}}
\paragraph{Mockery:} Also very elegant.\footnote{\url{https://github.com/mockery/mockery}}


\pagebreak
\section{Testing for Exceptions}

\subsection{Test for the Exception class only}
\begin{phpcode}
/**
 * @test
 * @expectedException \UnexpectedValueException
 */
public function createBreadWithNegativeSizeThrowsException()
{
    $this->subject->createBread(-1);
}
\end{phpcode}

\subsection{Test for the exception class, message and the code}
\begin{phpcode}
/**
 * @test
 * @expectedException \UnexpectedValueException
 * @expectedExceptionMessage size must be > 0.
 * @expectedExceptionCode 1323700434
 */
public function createBreadWithNegativeSizeThrowsException()
{
    $this->subject->createBread(-1);
}
\end{phpcode}


\pagebreak
\section{Testing abstract classes}

\subsection{Using the PHPUnit mock builder}

This will create an instance of the abstract class with all abstract methods mocked.\\

\begin{phpcode}
namespace OliverKlee\Coffee\Tests\Unit\Domain\Model;

use OliverKlee\Coffee\Domain\Model\AbstractBeverage;

class AbstractBeverageTest {
    /**
     * @var AbstractBeverage|\PHPUnit_Framework_MockObject_MockObject
     */
    protected $subject = null;

    protected function setUp()
    {
        $this->subject = $this->getMockForAbstractClass(
            AbstractBeverage::class
        );
    }
\end{phpcode}

\subsection{Creating a concrete subclass}
This is recommended if you need to provide your subclass with some additional or specific behavior.

In \texttt{Tests/Unit/Domain/Model/Fixtures/}, create a subclass of the abstract class:\\

\begin{phpcode}
namespace OliverKlee\Coffee\Tests\Unit\Domain\Model\Fixtures;

class TestingBeverage extends \OliverKlee\Coffee\Domain\Model\AbstractBeverage {
    // ...
}
\end{phpcode}

Then you can use and instantiate the concrete subclass in your unit tests:\\

\begin{phpcode}
use OliverKlee\Coffee\Tests\Unit\Domain\Model\Fixtures\TestingBeverage;

class AbstractBeverageTest {
    /**
     * @var TestingBeverage
     *
    protected $subject = null;

    protected function setUp()
    {
        $this->subject = new TestingBeverage();
    }
\end{phpcode}


\pagebreak
\section{Using the testing framework of the PHPUnit TYPO3 extension}

\begin{phpcode}
class DataMapperTest extends \Tx_Phpunit_TestCase {
    /**
     * @var \Tx_Phpunit_Framework
     */
    protected $testingFramework = null;

    protected $subject = null;

    protected function setUp()
    {
        $this->testingFramework = new \Tx_Phpunit_Framework('tx_oelib');

        $this->subject = new ...;
    }

    protected function tearDown()
    {
        $this->testingFramework->cleanUp();
    }

    /**
     * @test
     */
    public function findWithUidOfExistingRecordReturnsModelDataFromDatabase()
    {
        $title = 'foo';
        $uid = $this->testingFramework->createRecord(
            'tx_oelib_test', ['title' => $title]
        );

        self::assertSame($title, $this->subject->find($uid)->getTitle());
    }
\end{phpcode}

\subsection{Executable examples}

The functional tests for the FileUtility class in the tea example show what tests with vfsStream can look like.


\pagebreak
\section{Using mock file systems with vfsStream}
\subsection{Setting it all up}
\begin{phpcode}
use org\bovigo\vfs\vfsStream;
use org\bovigo\vfs\vfsStreamDirectory;

/**
 * @var \org\bovigo\vfs\vfsStreamFile
 */
protected $moreStuff;

protected function setUp()
{
    // This is the same as ::register and ::setRoot.
    $this->root = vfsStream::setup('home');
    $this->targetFilePath = vfsStream::url('home/target.txt');

    $this->subject = new ...
}
\end{phpcode}

\subsection{Using the files}
\small
\begin{phpcode}
/**
 * @test
 */
public function concatenateWithOneEmptySourceFileCreatesEmptyTargetFile()
{
    // This is one way to create a file with contents, using PHP's file functions.
    $sourceFileName = vfsStream::url('home/source.txt');
    // Just calling vfsStream::url does not create the file yet.
    // We need to write into it to create it.
    file_put_contents($sourceFileName, '');

    $this->subject->concatenate($this->targetFilePath, [$sourceFileName]);

    self::assertSame('', file_get_contents($this->targetFilePath));
}

/**
 * @test
 */
public function concatenateWithOneFileCopiesContentsFromSourceFileToTargetFile()
{
    // This is vfsStream's way of creating a file with contents.
    $contents = 'Hello world!';
    $sourceFileName = vfsStream::url('home/source.txt');
    vfsStream::newFile('source.txt')->at($this->root)->setContent($contents);

    $this->subject->concatenate($this->targetFilePath, [$sourceFileName]);

    self::assertSame($contents, file_get_contents($this->targetFilePath));
}
\end{phpcode}
\normalsize

\pagebreak
\section{PHPUnit assertions}
This list is current for PHPUnit 5.7.x.

\begin{verbatim}
assertArrayHasKey()
assertClassHasAttribute()
assertArraySubset()
assertClassHasStaticAttribute()
assertContains()
assertContainsOnly()
assertContainsOnlyInstancesOf()
assertCount()
assertDirectoryExists()
assertDirectoryIsReadable()
assertDirectoryIsWritable()
assertEmpty()
assertEqualXMLStructure()
assertEquals()
assertFalse()
assertFileEquals()
assertFileExists()
assertFileIsReadable()
assertFileIsWritable()
assertGreaterThan()
assertGreaterThanOrEqual()
assertInfinite()
assertInstanceOf()
assertInternalType()
assertIsReadable()
assertIsWritable()
assertJsonFileEqualsJsonFile()
assertJsonStringEqualsJsonFile()
assertJsonStringEqualsJsonString()
assertLessThan()
assertLessThanOrEqual()
assertNan()
assertNull()
assertObjectHasAttribute()
assertRegExp()
assertStringMatchesFormat()
assertStringMatchesFormatFile()
assertSame()
assertStringEndsWith()
assertStringEqualsFile()
assertStringStartsWith()
assertThat()
assertTrue()
assertXmlFileEqualsXmlFile()
assertXmlStringEqualsXmlFile()
assertXmlStringEqualsXmlString()
\end{verbatim}

\end{document}
