\documentclass[a4paper,10pt,headsepline]{scrartcl}
\usepackage[utf8]{inputenc}
\usepackage{a4wide,ae,hyperref,url}

\newcommand{\fett}[1]{\textsf{\textbf{#1}}}
\raggedbottom
\setcounter{secnumdepth}{3}

% Provide syntax highlighting for code
\newminted{php}{bgcolor=lightgrey,startinline=true,linenos=true}
\newminted{json}{bgcolor=lightgrey,startinline=false,linenos=true}
\newminted{bash}{bgcolor=lightgrey,startinline=false,linenos=true}

% Set of colors which look nice together
\definecolor{orange_light}{rgb}{1,0.8,0.4}
\definecolor{orange_med}{rgb}{0.753,0.62,0.373}
\definecolor{orange_dark}{rgb}{0.506,0.412,0.251}

\definecolor{green_light}{rgb}{0.8,1,0.4}
\definecolor{green_med}{rgb}{0.635,0.745,0.376}
\definecolor{green_dark}{rgb}{0.435,0.498,0.255}

\definecolor{blue_light}{rgb}{0.4,0.8,1}
\definecolor{blue_med}{rgb}{0.365,0.624,0.749}
\definecolor{blue_dark}{rgb}{0.251,0.42,0.502}

\definecolor{blue_cs3}{rgb}{0,.4,.8}

\definecolor{pink_light}{rgb}{1,0.435,0.812}
\definecolor{pink_med}{rgb}{0.745,0.38,0.62}
\definecolor{pink_dark}{rgb}{0.498,0.255,0.416}

\definecolor{yellow_light}{rgb}{1,1,0.4}
\definecolor{yellow_med}{rgb}{0.757,0.745,0.373}
\definecolor{yellow_dark}{rgb}{0.506,0.49,0.251}

\definecolor{lightgrey}{rgb}{.9,.9,.9}

% blue (for URLs)
\definecolor{blue}{rgb}{0,0,1}


\usepackage[ngerman]{babel}
\selectlanguage{ngerman}

\hypersetup{
 pdfauthor={Oliver Klee, typo3-coding@oliverklee.de},
 pdftitle={Testgetriebene Entwicklung mit PHPUnit (mit oder ohne TYPO3)},
 pdfsubject={Handout zu Oliver Klees Workshops zu testgetriebener Entwicklung mit PHPUnit (mit oder ohne TYPO3).},
 pdfkeywords={TDD, unit testing, PHPUnit, workshop, seminar, PHP}
}

\author{
  Oliver Klee | \texttt{typo3-coding@oliverklee.de} | \texttt{@oliklee} \\
  \url{https://github.com/oliverklee/tdd-reader}
}

\date{Version 2.0.2, \today, für PHP >= 5.6 und TYPO3 CMS 7.6}

\title{
  Spickzettel zu testgetriebener Entwicklung mit PHPUnit
}

\begin{document}

\maketitle

\section*{Lizenz}

Dieser Reader ist unter einer \emph{Creative-Commons}-Lizenz lizensiert, und zwar konkret unter der \emph{Namensnennung-Weitergabe unter gleichen Bedingungen~4.0 (CC BY-SA 4.0)}. Das bedeutet, dass ihr den Reader unter diesen Bedingungen für euch kostenlos verbreiten, bearbeiten und nutzen könnt (auch kommerziell):

\begin{description}
  \item[Namensnennung.] Ihr müsst den Namen des Autors (Oliver Klee) nennen. Wenn ihr dabei zusätzlich auch noch die Quelle\footnote{\url{https://github.com/oliverklee/tdd-reader}} nennt, wäre das nett. Und wenn ihr mir zusätzlich eine Freude machen möchtet, sagt mir per E-Mail Bescheid.
  \item[Weitergabe unter gleichen Bedingungen.] Wenn ihr diesen Inhalt bearbeitet oder in anderer Weise umgestaltet, verändert oder als Grundlage für einen anderen Inhalt verwendet, dann dürft ihr den neu entstandenen Inhalt nur unter Verwendung identischer Lizenzbedingungen weitergeben.
  \item[Lizenz nennen.] Wenn ihr den Reader weiter verbreitet, müsst ihr dabei auch die Lizenzbedingungen nennen oder beifügen.
\end{description}

Die ausführliche Version dieser Lizenz findet ihr online.\footnote{\url{http://creativecommons.org/licenses/by-sa/4.0/}}


\pagebreak

\tableofcontents

\pagebreak

\section{Benennung von Dateien und Klassen}

\subsection{Dateinamen}

\begin{tabular}{|l|l|}
  \hline
  \fett{Dateiname des Produktionscodes} & \fett{Name der Testdatei} \\
  \hline
  \texttt{Classes/Domain/Model/Shoe.php} & \texttt{Tests/Unit/Domain/Model/ShoeTest.php} \\
  \hline
  \texttt{Classes/Service/BaristaService.php} & \texttt{Tests/Unit/Service/BaristaServiceTest.php} \\
  \hline
\end{tabular}


\subsection{Klassennamen}

\small
\begin{tabular}{|l|l|}
  \hline
  \fett{Name der Klasse im Produktionscode} & \fett{Name der Testklasse} \\
  \hline
  \texttt{Shoes\textbackslash Shop\textbackslash Domain\textbackslash Model\textbackslash Shoe} & \texttt{Shoes\textbackslash Shop\textbackslash Tests\textbackslash Unit\textbackslash Domain\textbackslash Model\textbackslash ShoeTest} \\
  \hline
  \texttt{Shoes\textbackslash Shop\textbackslash Service\textbackslash BaristaService} & \texttt{Shoes\textbackslash Shop\textbackslash Tests\textbackslash Unit\textbackslash Service\textbackslash BaristaServiceTest} \\
  \hline
\end{tabular}
\normalsize

\pagebreak
\section{Struktur von Testklassen}

\subsection{Extbase-Extensions}

Es gibt auf GitHub dazu auch ein Beispielprojekt (das \emph{Tea-Example}):\\
\url{https://github.com/oliverklee/ext_tea}\\

\begin{phpcode}
namespace \OliverKlee\Shop\Tests\Unit\Domain\Model;

use OliverKlee\Shop\Domain\Model\Article;

class ArticleTest extends \TYPO3\CMS\Core\Tests\UnitTestCase {
    /**
     * @var Article;
     */
    protected $subject = null;

    protected function setUp()
    {
        $this->subject = new Article;
        $this->subject->initializeObject();
    }

    /**
     * @test
     */
    public function getNameInitiallyReturnsEmptyString()
    {
        self::assertSame('', $this->subject->getName());
    }

    /**
     * @test
     */
    public function setNameSetsName()
    {
        $name = 'foo bar';

        $this->subject->setName($name);

        self::assertSame($name, $this->subject->getName());
    }

    // ...
}
\end{phpcode}

\subsection{Nicht-Extbase-Extensions}

\begin{phpcode}
class AttachmentTest extends \Tx_Phpunit_TestCase {
    /**
     * @var \Tx_Oelib_Attachment
     */
    protected $subject = null;

    protected function setUp()
    {
        $this->subject = new \Tx_Oelib_Attachment();
    }

    /**
     * @test
     */
    public function getFileNameInitiallyReturnsAnEmptyString()
    {
        self::assertSame('', $this->subject->getFileName());
    }

    /**
     * @test
     */
    public function getFileNameWithFileNameSetReturnsFileName()
    {
        $fileName = 'test.txt';

        $this->subject->setFileName($fileName);

        self::assertSame($fileName, $this->subject->getFileName());
    }

    /**
     * @test
     * @expectedException \UnexpectedValueException
     */
    public function setFileNameWithEmptyFileNameThrowsException()
    {
        $this->subject->setFileName('');
    }

    // ...
}
\end{phpcode}

\subsection{Nicht-TYPO3-PHP-Project mit Composer}

Es gibt auf GitHub dazu auch ein leeres Startprojekt:\\
\url{https://github.com/oliverklee/tdd-seed}

\subsubsection{composer.json}

Diese composer.json installiert PHPUnit und vfsStream für \fett{PHP bis 5.6}:\\

\begin{jsoncode}
{
    "require-dev": {
        "phpunit/phpunit": "^5.7.19",
        "mikey179/vfsStream": "^1.6.4"
    },
    "autoload": {
        "psr-4": {
            "..."
        }
    },
    "autoload-dev": {
        "psr-4": {
            "..."
        }
    }
}
\end{jsoncode}

Diese composer.json installiert PHPUnit und vfsStream für \fett{PHP ab 6}:\\

\begin{jsoncode}
{
    "require-dev": {
        "phpunit/phpunit": "^6.0.13",
        "mikey179/vfsStream": "^1.6.4"
    },
    "autoload": {
        "psr-4": {
            "..."
        }
    },
    "autoload-dev": {
        "psr-4": {
            "..."
        }
    }
}
\end{jsoncode}

\subsubsection{Testcase}
\begin{phpcode}
namespace OliverKlee\Books\Tests\Unit\Domain\Model;

use OliverKlee\Books\Domain\Model;

class BookTest extends \PHPUnit_Framework_TestCase {
    /**
     * @var Book
     */
    protected $subject = null;

    protected function setUp()
    {
        $this->subject = new Book();
    }

    /**
     * @test
     */
    public function getTitleInitiallyReturnsEmptyString()
    {
        self::assertSame('', $this->subject->getTitle());
    }

    /**
     * @test
     */
    public function setTitleSetsTitle()
    {
        $title = 'foo bar';

        $this->subject->setTitle($title);

        self::assertSame($title, $this->subject->getTitle());
    }
}
\end{phpcode}


\pagebreak
\section{Tests ausführen}

\subsection{Nicht-TYPO3-Projekte}

\subsubsection{Auf der Kommandozeile}

\begin{bashcode}
vendor/bin/phpunit Test/
\end{bashcode}

\subsubsection{In PhpStorm}
\begin{enumerate}
  \item Settings > Languages \& Frameworks > PHP > PHPUnit
  \item PHPUnit library > Use Composer autoloader
  \item PHPUnit library > Path to script: \texttt{vendor/autoload}
  \item OK
  \item auf den Order \texttt{Tests/} rechtsklicken (oder einen anderen Ordner oder eine Testdatei)
  \item Run 'Tests'
\end{enumerate}


\subsection{TYPO3-Extensions}

\subsubsection{In PhpStorm}

\paragraph{Für eine existierende TYPO3-Installation im Composer-Modus:}

Bei diesem Ansatz werden alle installierten Extensions geladen, sodass ihr auch die Features der PHPUnit-Extension nutzen könnt.

\begin{enumerate}
  \item Settings > Languages \& Frameworks > PHP > PHPUnit
  \item PHPUnit library > Use Composer autoloader
  \item PHPUnit library > Path to script: \texttt{vendor/autoload} aus dem Document-Root der TYPO3-Installation
  \item Test runner > Default configuration file: \texttt{typo3/sysext/core/Build/UnitTests.xml} innerhalb des Document-Roots der TYPO3-Installation
  \item OK
  \item Run > Edit Configurations
  \item Defaults > PHPUnit
  \item Command Line > Environment variables
  \item zwei Variablen hinzufügen:
    \begin{itemize}
      \item \texttt{TYPO3\_CONTENT = Development}
      \item \texttt{TYPO3\_PATH\_WEB =} der absolute Pfad zum Document-Root der TYPO3-Installation (ohne den Slash am Ende)
    \end{itemize}
  \item auf den Order \texttt{Tests/} rechtsklicken (oder einen anderen Ordner oder eine Testdatei)
  \item Run 'Tests'
\end{enumerate}


\paragraph{Für eine existierende TYPO3-Installation im Klassik-Modus (Nicht-Composer-Modus):}

In diesem Fall werdet ihr keine Klassen aus anderen Extensions autoloaden können, d.\,h., ihr werdet auch keine Features der PHPUnit-Extension nutzen können (und auch nicht von anderen Extension-Abhängigkeiten).

\begin{enumerate}
  \item Wenn ihr den TYPO3-Source per git statt als TAR-Paket heruntergeladen habt, braucht ihr ein \texttt{composer install} im TYPO3-Source-Verzeichnis.
  \item Settings > Languages \& Frameworks > PHP > PHPUnit
  \item PHPUnit library > Use Composer autoloader
  \item PHPUnit library > Path to script: \texttt{vendor/autoload} innerhalb des TYPO3-Source
  \item Test runner > Default configuration file: \texttt{typo3/sysext/core/Build/UnitTests.xml} innerhalb des TYPO3-Document-Roots
  \item OK
  \item Run > Edit Configurations
  \item Defaults > PHPUnit
  \item Command Line > Environment variables
  \item zwei Variablen hinzufügen:
    \begin{itemize}
      \item \texttt{TYPO3\_CONTENT = Development}
      \item \texttt{TYPO3\_PATH\_WEB =} der absolute Pfad zum Document-Root der TYPO3-Installation (ohne den Slash am Ende)
    \end{itemize}
  \item auf den Order \texttt{Tests/} rechtsklicken (oder einen anderen Ordner oder eine Testdatei)
  \item Run 'Tests'
\end{enumerate}


\paragraph{Ohne eine existierende TYPO3-Installation}

Dieser Ansatz benutzt das TYPO3-Extension-Skelett\footnote{\url{https://github.com/helhum/ext_scaffold}} von Helmut Hummel und Nicole Cordes.

Fügt die folgenden Abschnitte zur \texttt{composer.json} eurer Extension hinzu:

\small
\begin{jsoncode}
"require": {
  "typo3/cms": "~7.6.0"
},
"require-dev": {
  "namelesscoder/typo3-repository-client": "^1.2",
  "nimut/testing-framework": "^1.0",
  "mikey179/vfsStream": "^1.4",
  "phpunit/phpunit": "^4.7 || ^5.0"
},
"config": {
  "vendor-dir": ".Build/vendor",
  "bin-dir": ".Build/bin"
},
"scripts": {
  "post-autoload-dump": [
    "mkdir -p .Build/Web/typo3conf/ext/",
    "[ -L .Build/Web/typo3conf/ext/tea ] || ln -snvf ../../../../. .Build/Web/typo3conf/ext/tea"
  ]
},
"extra": {
  "typo3/cms": {
    "cms-package-dir": "{$vendor-dir}/typo3/cms",
    "web-dir": ".Build/Web"
  }
}
\end{jsoncode}
\normalsize

Ersetzt dabei \texttt{tea} in Zeile 18 durch den Schlüssel eurer Extension.

Wenn ihr außerdem noch andere, im TER verfügbarer Extensions in den Tests nutzen möchtet (z.\,B.~die PHPUnit-Extension), fügt noch diese Abschnitte zu eurer \texttt{composer.json} hinzu (bzw.~ mergt sie):

\begin{jsoncode}
"repositories": [
{
  "type": "composer",
  "url": "https://composer.typo3.org/"
}
],
"require-dev": {
  "typo3-ter/phpunit": "*"
},
\end{jsoncode}

Führt danach diese Schritte in PhpStorm aus:
\begin{enumerate}
  \item Settings > Languages \& Frameworks > PHP > PHPUnit
  \item PHPUnit library > Use Composer autoloader
  \item PHPUnit library > Path to script:
    \begin{itemize}
      \item auf den Button \emph{Show hidden files and directories} klicken
      \texttt{.Build/vendor/autoload.php} im Verzeichnis der Extension
    \end{itemize}
  \item Test runner > Default configuration file: \texttt{typo3/sysext/core/Build/UnitTests.xml} im TYPO3-Source-Verzeichnis
  \item OK
  \item Run > Edit Configurations
  \item Defaults > PHPUnit
  \item Command Line > Environment variables
  \item zwei Variablen hinzufügen:
    \begin{itemize}
      \item \texttt{TYPO3\_CONTENT = Development}
      \item \texttt{TYPO3\_PATH\_WEB =} der absolute Pfad zu \texttt{.Build/Web} im Verzeichnis der Extension (ohne den Slash am Ende)
    \end{itemize}
  \item auf den Order \texttt{Tests/} rechtsklicken (oder einen anderen Ordner oder eine Testdatei)
  \item Run 'Tests'
\end{enumerate}



\subsubsection{Mit dem Backend-Modul der PHPUnit-Extension}

Dieser Ansatz funktioniert sowohl für TYPO3-Installationen im Composer-Modus als auch im Klassik-Modus.

Dabei werden alle installierten Extensions geladen (inklusive der PHPUnit-Extension), sodass ihr die Features der PHPUnit-Extension nutzen könnt.

Allerdings werden die Tests dann im Kontext der aktuellen Backend-Session ausgeführt, was die Tests sehr zehrbrechlich macht. Für die meisten Unit-Tests von Extensions funktioniert das, für funktionale Tests wird es hingegen meistens nicht git funktionieren.

\begin{enumerate}
  \item Admin > PHPUnit
\end{enumerate}



\pagebreak
\section{Mocks}

\subsection{Warum mocken?}
\begin{itemize}
  \item um eine Methode \glqq auszuschalten\grqq\ (damit sie nicht in die DB schreibt, kein Cruise-Missile abschießt etc.) und \texttt{null} zurückzugeben
  \item um einer Methode einen bestimmten Rückgabewert zu geben oder sie eine Exception werfen zu lassen
  \item um zu testen, dass eine Methode auf eine bestimmte Art und Weise aufgerufen wird
\end{itemize}

\subsection{Tools für Mocks}

\paragraph{Prophecy:} Das empfohlene, einfach benutzbare, aktuelle Mocking-Framework. Allerdings kann es keine partiellen Mocks erzeugen.\footnote{Prophecy-Cheatsheet:\\ \url{https://github.com/oliverklee/tdd-reader/blob/master/AdditionalDocuments/prophecy-cheatsheet.pdf}}
\paragraph{PHPUnit mocks:} Die alte Art, Mocks zu erzeugen. Mocking ist damit etwas unhandlich, aber dafür kann es auch partielle Mocks erzeugen.\footnote{PHPUnit-Mocking-Cheatsheet:\\ \url{https://github.com/oliverklee/tdd-reader/blob/master/AdditionalDocuments/mocking-cheatsheet.pdf}}
\paragraph{Mockery:} Auch sehr elegant.\footnote{\url{https://github.com/mockery/mockery}}



\pagebreak
\section{Auf Exceptions testen}

\subsection{Nur auf die Klasse testen}
\begin{phpcode}
/**
 * @test
 * @expectedException \UnexpectedValueException
 */
public function createBreadWithNegativeSizeThrowsException()
{
    $this->subject->createBread(-1);
}
\end{phpcode}

\subsection{Auf Klasse, Nachricht und Code testen}
\begin{phpcode}
/**
 * @test
 * @expectedException \UnexpectedValueException
 * @expectedExceptionMessage size must be > 0.
 * @expectedExceptionCode 1323700434
 */
public function createBreadWithNegativeSizeThrowsException()
{
    $this->subject->createBread(-1);
}
\end{phpcode}


\pagebreak
\section{Abstrakte Klassen testen}

\subsection{Den PHPUnit-Mock-Builder benutzen}

Dies erzeugt eine Instanz der abstrakten Klassen, wobei alle abstrakten Methoden gemockt werden.\\

\begin{phpcode}
namespace OliverKlee\Coffee\Tests\Unit\Domain\Model;

use OliverKlee\Coffee\Domain\Model\AbstractBeverage;

class AbstractBeverageTest {
    /**
     * @var AbstractBeverage|\PHPUnit_Framework_MockObject_MockObject
     */
    protected $subject = null;

    protected function setUp()
    {
        $this->subject = $this->getMockForAbstractClass(
            AbstractBeverage::class
        );
    }
\end{phpcode}

\subsection{Eine konkrete Unterklasse erstellen}

Dies wird empfohlen, wenn die Subklasse zusätzliches oder spezifisches Verhalten haben soll.

Erzeugt in \texttt{Tests/Unit/Unit/Domain/Model/Fixtures/} eine Unterklasse der abstrakten Klasse:\\

\begin{phpcode}
namespace OliverKlee\Coffee\Tests\Unit\Domain\Model\Fixtures;

class TestingBeverage extends \OliverKlee\Coffee\Domain\Model\AbstractBeverage {
    // ...
}
\end{phpcode}

Dann könnt ihr die konkrete Unterklasse in euren Unit-Tests usen und instanziieren.\\

\begin{phpcode}
use OliverKlee\Coffee\Tests\Unit\Domain\Model\Fixtures\TestingBeverage;

class AbstractBeverageTest {
    /**
     * @var TestingBeverage
     *
    protected $subject = null;

    protected function setUp()
    {
        $this->subject = new TestingBeverage();
    }
\end{phpcode}


\pagebreak
\section{Das Test-Framework der PHPUnit-TYPO3-Extension benutzen}

\begin{phpcode}
class DataMapperTest extends \Tx_Phpunit_TestCase {
    /**
     * @var \Tx_Phpunit_Framework
     */
    protected $testingFramework = null;

    protected $subject = null;

    protected function setUp()
    {
        $this->testingFramework = new \Tx_Phpunit_Framework('tx_oelib');

        $this->subject = new ...;
    }

    protected function tearDown()
    {
        $this->testingFramework->cleanUp();
    }

    /**
     * @test
     */
    public function findWithUidOfExistingRecordReturnsModelDataFromDatabase()
    {
        $title = 'foo';
        $uid = $this->testingFramework->createRecord(
            'tx_oelib_test', ['title' => $title]
        );

        self::assertSame($title, $this->subject->find($uid)->getTitle());
    }
\end{phpcode}

\pagebreak
\section{Gemockte Dateisystem mit vfsStream benutzen}

\subsection{Lauffähige Beispiele}

Die funktionalen Tests zur FileUtility-Klasse im Tea-Beispiel zeigen, wie Tests mit vfsStream aussehen können.


\subsection{Einrichten}
\begin{phpcode}
use org\bovigo\vfs\vfsStream;
use org\bovigo\vfs\vfsStreamDirectory;

/**
 * @var \org\bovigo\vfs\vfsStreamFile
 */
protected $moreStuff;

protected function setUp()
{
    // This is the same as ::register and ::setRoot.
    $this->root = vfsStream::setup('home');
    $this->targetFilePath = vfsStream::url('home/target.txt');

    $this->subject = new ...
}
\end{phpcode}

\subsection{Die Dateien benutzen}
\small
\begin{phpcode}
/**
 * @test
 */
public function concatenateWithOneEmptySourceFileCreatesEmptyTargetFile()
{
    // This is one way to create a file with contents, using PHP's file functions.
    $sourceFileName = vfsStream::url('home/source.txt');
    // Just calling vfsStream::url does not create the file yet.
    // We need to write into it to create it.
    file_put_contents($sourceFileName, '');

    $this->subject->concatenate($this->targetFilePath, [$sourceFileName]);

    self::assertSame('', file_get_contents($this->targetFilePath));
}

/**
 * @test
 */
public function concatenateWithOneFileCopiesContentsFromSourceFileToTargetFile()
{
    // This is vfsStream's way of creating a file with contents.
    $contents = 'Hello world!';
    $sourceFileName = vfsStream::url('home/source.txt');
    vfsStream::newFile('source.txt')->at($this->root)->setContent($contents);

    $this->subject->concatenate($this->targetFilePath, [$sourceFileName]);

    self::assertSame($contents, file_get_contents($this->targetFilePath));
}
\end{phpcode}
\normalsize


\pagebreak
\section{PHPUnit-Assertions}
Diese Liste ist aktuell für PHPUnit 5.7.x.

\begin{verbatim}
assertArrayHasKey()
assertClassHasAttribute()
assertArraySubset()
assertClassHasStaticAttribute()
assertContains()
assertContainsOnly()
assertContainsOnlyInstancesOf()
assertCount()
assertDirectoryExists()
assertDirectoryIsReadable()
assertDirectoryIsWritable()
assertEmpty()
assertEqualXMLStructure()
assertEquals()
assertFalse()
assertFileEquals()
assertFileExists()
assertFileIsReadable()
assertFileIsWritable()
assertGreaterThan()
assertGreaterThanOrEqual()
assertInfinite()
assertInstanceOf()
assertInternalType()
assertIsReadable()
assertIsWritable()
assertJsonFileEqualsJsonFile()
assertJsonStringEqualsJsonFile()
assertJsonStringEqualsJsonString()
assertLessThan()
assertLessThanOrEqual()
assertNan()
assertNull()
assertObjectHasAttribute()
assertRegExp()
assertStringMatchesFormat()
assertStringMatchesFormatFile()
assertSame()
assertStringEndsWith()
assertStringEqualsFile()
assertStringStartsWith()
assertThat()
assertTrue()
assertXmlFileEqualsXmlFile()
assertXmlStringEqualsXmlFile()
assertXmlStringEqualsXmlString()
\end{verbatim}

\end{document}
